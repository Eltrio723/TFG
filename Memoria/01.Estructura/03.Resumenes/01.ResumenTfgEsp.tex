%%%%%%%%%%%%%%%%%%%%%%%%%%%%%%%%%%
% Página de resumen del proyecto %
%%%%%%%%%%%%%%%%%%%%%%%%%%%%%%%%%%
% !TEX root = A0.TFG.tex

\pagestyle{fancy}
\renewcommand{\headrulewidth}{0pt}
\addstarredchapter{Resumen}

\begin{center}
	\scshape
	E.T.S. de Ingenierías Informática y de Telecomunicación, Universidad de Granada
\end{center}

\bigskip

\begin{center}
	\Large \scshape
	\textbf{\tfgtitlename}
\end{center}

\bigskip \bigskip \bigskip

\begin{minipage}{\textwidth}

\textbf{Autor:} \tfgauthorname

\medskip

\textbf{Tutores:} \tfgtutornameA , \tfgtutornameB

\medskip

%\textbf{Cotutor:} <Nombre del cotutor>\ (elimina esta línea si no hay cotutor)

\medskip

\textbf{Departamento:} Lenguajes y Sistemas Informáticos

\medskip

\textbf{Titulación:} Grado en Ingeniería Informática

\medskip

\textbf{Palabras clave:} Realidad virtual, entrenamiento cognitivo, deterioro cognitivo, memoria, percepción, razonamiento, mayores, videojuegos, juego serio, Unity, Oculus.

\bigskip \bigskip


\end{minipage}

\begin{center}
	\textbf{Resumen}
\end{center}





Este proyecto se ha desarrollado con el objetivo de proponer y analizar el funcionamiento de la tecnología de realidad virtual como herramienta para el tratamiento del deterioro cognitivo leve así como el entrenamiento de las habilidades cognitivas básicas como la memoria, la percepción y el razonamiento. Con dicho objetivo se ha estudiado la cognición humana y la realidad virtual para desarrollar una demostración de juego serio que pueda ser utilizado como apoyo a los tratamientos de rehabilitación cognitiva en personas que la hayan visto deteriorada, o como entrenamiento para mantener una cognición sana especialmente en etapas de edad avanzada. Se intenta que la realidad virtual sirva de aliciente y motivación para las personas a la vez que se ofrece la posibilidad de mejorar la eficacia de los tratamientos actuales. Tras el desarrollo del proyecto y posterior prueba con personas, se evalúa la usabilidad del sistema y se obtienen buenos resultados. Los distintos ejercicios cognitivos han tenido buen impacto y personas de edad avanzada no han tenido problema en adaptarse a la realidad virtual y manejar el juego por sí mismas. 




\blankpage
