%%%%%%%%%%%%%%%%%%%%%%%%%%%%%%%%%%%%%%%%%%%
% Página de resumen del proyecto en inglés%
%%%%%%%%%%%%%%%%%%%%%%%%%%%%%%%%%%%%%%%%%%%
% !TEX root = A0.TFG.tex

\pagestyle{fancy}
\addstarredchapter{Abstract}

\begin{center}
	\scshape
	E.T.S. de Ingenierías Informática y de Telecomunicación, Universidad de Granada
\end{center}

\bigskip

\begin{center}
	\Large \scshape
	\textbf{\tfgtitlenameENG}
\end{center}

\bigskip \bigskip \bigskip

\begin{minipage}{\textwidth}

\textbf{Author:} \tfgauthorname

\medskip

\textbf{Supervisors:} \tfgtutornameA , \tfgtutornameB

\medskip

%\textbf{Co-supervisor:} <Nombre del cotutor>\ (elimina esta línea si no hay cotutor)

\medskip

\textbf{Department:} Lenguajes y Sistemas Informáticos

\medskip

\textbf{Degree:} Grado en Ingeniería Informática

\medskip

\textbf{Keywords:} Virtual reality, brain training, mild cognitive impairment, memory, perception, reasoning, elderly, videogames, serious game, Unity, Oculus.

\bigskip \bigskip


\end{minipage}

\begin{center}
	\textbf{Abstract}
\end{center}



This proyect has been developed with the aim of proposing and analising the use of virtual reality technology as a tool in the treatment of mild cognitive impairment as well as the training of basic cognitive skills such as memory, perception and reasoning. With this aim, human cognition and virtual reality have been studied in order to develop a demo of a serious game able to be used as support for cognitive rehabilitation treatments on people who have seen their cognition deteriorated, or as regular training in order to keep a healthy one specially at older ages. It is the intention that virtual reality aid at keeping the motivation for the people as well as allowing the posibility to improve the eficacy of current treatments. After the development of this proyect and later test with people, the system's usability is evaluated.

\todo[inline, size=\tiny]{Añadir conclusiones evaluación al resumen en inglés}

\url{https://github.com/Eltrio723/TFG}

\blankpage
