

\pagestyle{fancy}
\fancyhead[LE,RO]{\thepage}
\fancyhead[RE]{Apéndice} %
\fancyhead[LO]{\nouppercase{\rightmark}}

\chapterbegin{Gestión del Proyecto}
\label{sec:apendice:Proyecto}
\minitoc



%\minitoc

\section{Gestión del proyecto}
Durante el desarrollo de este proyecto se ha utilizado Git como herramienta de control de versiones, creando un repositorio en Github para almacenar todos los archivos.

El repositorio se encuentra en la siguiente URL: \url{https://github.com/Eltrio723/TFG}

Los contenidos del repositorio se dividen en tres partes: 
En la carpeta Memoria está todo lo referente a esta memoria, la cual se ha realizado en Latex y puede ser recompilada. En la ruta \url{https://github.com/Eltrio723/TFG/blob/master/Memoria/00.Auxiliar/01.Output/TrabajoFinGrado.pdf} se encuentra la última compilación de este PDF.

En la carpeta Proyecto están todos los archivos de Unity necesarios para ejecutar el juego en cualquier ordenador usando la versión 2023.1.9f1 del motor. Además es necesario disponer de unas Meta Quest 2 con una cuenta de desarrollador, el modo desarrollador activo, e instalar en un pc la aplicación de escritorio de Oculus así como la aplicación Meta Quest Developer Hub.

Finalmente en la carpeta Recursos se encuentran algunos archivos que han servido de ayuda durante el desarrollo de este proyecto.


Además, el repositorio contiene una \textit{release} llamada Entrega TFG, dónde se añaden varios vídeos de demostración del juego y de las pruebas realizadas con personas.

\chapterend
