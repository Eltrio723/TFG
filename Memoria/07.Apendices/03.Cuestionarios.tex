%%%%%%%%%%%%%%%%%%%%%%%%%%%%%%%%%%%%%%%%%%%%%%%%%%%%%%%%%%%%%%%%%%%
%%% Documento LaTeX 																						%%%
%%%%%%%%%%%%%%%%%%%%%%%%%%%%%%%%%%%%%%%%%%%%%%%%%%%%%%%%%%%%%%%%%%%
% Título:		Apéndice A
% Autor:  	Ignacio Moreno Doblas
% Fecha:  	2014-02-01, actualizado 2019-11-11
% Versión:	0.5.0
%%%%%%%%%%%%%%%%%%%%%%%%%%%%%%%%%%%%%%%%%%%%%%%%%%%%%%%%%%%%%%%%%%%%

\pagestyle{fancy}
\fancyhead[LE,RO]{\thepage}
\fancyhead[RE]{Apéndice} %
\fancyhead[LO]{\nouppercase{\rightmark}}

\chapterbegin{Cuestionarios}
\label{sec:apendice:Custionarios}

\todo[inline, size=\tiny]{Reescribir preguntas formularios. Rellenar respuestas cuestionarios}

%\minitoc

\section{Plantilla}

Este cuestionario consta de 10 preguntas para evaluar la eficacia, eficiencia y satisfacción del sistema que acabas de probar. 

\paragraph{1.} Creo que me gustaría utilizar este sistema con frecuencia.

% Please add the following required packages to your document preamble:
% \usepackage{graphicx}
\begin{table}[H]
	\centering
	\resizebox{\textwidth}{!}{%
		\begin{tabular}{|l|l|l|l|l|}
			\hline
			Totalmente en desacuerdo & En desacuerdo & Neutro & De acuerdo & Totalmente de acuerdo \\ \hline
			&               &        &            &                       \\ \hline
		\end{tabular}%
	}
\end{table}

\paragraph{2.} Encontré el sistema innecesariamente complejo.

\begin{table}[H]
	\centering
	\resizebox{\textwidth}{!}{%
		\begin{tabular}{|l|l|l|l|l|}
			\hline
			Totalmente en desacuerdo & En desacuerdo & Neutro & De acuerdo & Totalmente de acuerdo \\ \hline
			&               &        &            &                       \\ \hline
		\end{tabular}%
	}
\end{table}

\paragraph{3.} Pensé que el sistema era fácil de usar.

\begin{table}[H]
	\centering
	\resizebox{\textwidth}{!}{%
		\begin{tabular}{|l|l|l|l|l|}
			\hline
			Totalmente en desacuerdo & En desacuerdo & Neutro & De acuerdo & Totalmente de acuerdo \\ \hline
			&               &        &            &                       \\ \hline
		\end{tabular}%
	}
\end{table}

\paragraph{4.} Creo que necesitaría el apoyo de un técnico para poder utilizar este sistema.

\begin{table}[H]
	\centering
	\resizebox{\textwidth}{!}{%
		\begin{tabular}{|l|l|l|l|l|}
			\hline
			Totalmente en desacuerdo & En desacuerdo & Neutro & De acuerdo & Totalmente de acuerdo \\ \hline
			&               &        &            &                       \\ \hline
		\end{tabular}%
	}
\end{table}

\paragraph{5.} Encontré que las diversas funciones de este sistema estaban bien integradas.

\begin{table}[H]
	\centering
	\resizebox{\textwidth}{!}{%
		\begin{tabular}{|l|l|l|l|l|}
			\hline
			Totalmente en desacuerdo & En desacuerdo & Neutro & De acuerdo & Totalmente de acuerdo \\ \hline
			&               &        &            &                       \\ \hline
		\end{tabular}%
	}
\end{table}

\paragraph{6.} Pensé que había demasiada inconsistencia en este sistema.

\begin{table}[H]
	\centering
	\resizebox{\textwidth}{!}{%
		\begin{tabular}{|l|l|l|l|l|}
			\hline
			Totalmente en desacuerdo & En desacuerdo & Neutro & De acuerdo & Totalmente de acuerdo \\ \hline
			&               &        &            &                       \\ \hline
		\end{tabular}%
	}
\end{table}

\paragraph{7.} Me imagino que la mayoría de la gente aprendería a utilizar este sistema muy rápidamente.

\begin{table}[H]
	\centering
	\resizebox{\textwidth}{!}{%
		\begin{tabular}{|l|l|l|l|l|}
			\hline
			Totalmente en desacuerdo & En desacuerdo & Neutro & De acuerdo & Totalmente de acuerdo \\ \hline
			&               &        &            &                       \\ \hline
		\end{tabular}%
	}
\end{table}

\paragraph{8.} Encontré el sistema muy complicado de usar.

\begin{table}[H]
	\centering
	\resizebox{\textwidth}{!}{%
		\begin{tabular}{|l|l|l|l|l|}
			\hline
			Totalmente en desacuerdo & En desacuerdo & Neutro & De acuerdo & Totalmente de acuerdo \\ \hline
			&               &        &            &                       \\ \hline
		\end{tabular}%
	}
\end{table}

\paragraph{9.} Me sentí muy seguro usando el sistema.

\begin{table}[H]
	\centering
	\resizebox{\textwidth}{!}{%
		\begin{tabular}{|l|l|l|l|l|}
			\hline
			Totalmente en desacuerdo & En desacuerdo & Neutro & De acuerdo & Totalmente de acuerdo \\ \hline
			&               &        &            &                       \\ \hline
		\end{tabular}%
	}
\end{table}

\paragraph{10.} Necesitaba aprender muchas cosas antes de empezar con este sistema.

\begin{table}[H]
	\centering
	\resizebox{\textwidth}{!}{%
		\begin{tabular}{|l|l|l|l|l|}
			\hline
			Totalmente en desacuerdo & En desacuerdo & Neutro & De acuerdo & Totalmente de acuerdo \\ \hline
			&               &        &            &                       \\ \hline
		\end{tabular}%
	}
\end{table}



\newpage

\section{Resultados usabilidad}


Tras presentar este cuestionario a X personas, se han obtenido los resultados de usabilidad mostrados en la tabla \ref{tab:resultados_usabilidad}. Obteniendo una puntuación media en la escala de usabilidad de X. Véase sección \ref{sec:evaluacion} para más información sobre esta puntuación.


\begin{table}[H]
	\centering
	\resizebox{\textwidth}{!}{%
		\begin{tabular}{|l|l|l|l|l|l|l|l|l|l|l|l|l|l|}
			\hline
			Encuestado & Preg. 1 & Preg. 2 & Preg. 3 & Preg. 4 & Preg. 5 & Preg. 6 & Preg. 7 & Preg. 8 & Preg. 9 & Preg. 10 & Suma impares & Suma pares & Total SUS \\ \hline
			A & 4 & 3 & 4 & 4 & 4 & 2 & 4 & 2 & 2 & 3 & 18 & 14 & 60 \\ \hline
			B & 5 & 1 & 5 & 4 & 5 & 2 & 5 & 1 & 5 & 2 & 25 & 10 & 87.5 \\ \hline
			C & 4 & 1 & 4 & 3 & 4 & 2 & 4 & 2 & 3 & 2 & 19 & 10 & 72.5 \\ \hline
			D & 4 & 1 & 5 & 2 & 4 & 2 & 4 & 2 & 4 & 1 & 21 & 8 & 82.5 \\ \hline
			E & 4 & 2 & 4 & 4 & 4 & 2 & 5 & 2 & 3 & 3 & 20 & 13 & 67.5 \\ \hline
			F & 0 & 0 & 0 & 0 & 0 & 0 & 0 & 0 & 0 & 0 & 0 & 0 & 0 \\ \hline
			G & 0 & 0 & 0 & 0 & 0 & 0 & 0 & 0 & 0 & 0 & 0 & 0 & 0 \\ \hline
		\end{tabular}%
	}
	\caption{Resultados de usabilidad obtenidos tras la prueba}
	\label{tab:resultados_usabilidad}
\end{table}


\chapterend
