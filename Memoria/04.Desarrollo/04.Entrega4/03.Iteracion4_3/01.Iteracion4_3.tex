
\subsection{Iteración 3}


En esta última iteración se busca que un usuario real realice las distintas pruebas para poder evaluarlas, encontrar posibles problemas y solucionarlos. Para ello se cuenta con una persona joven, con conocimientos tecnológicos básicos. Esta persona ha repetido varias veces cada una de las pruebas creadas en este proyecto y de su experiencia se pueden sacar las siguientes conclusiones sobre las pruebas:


\begin{itemize}
    \item {Durante las pruebas de baile y figuras, en las que el jugador debe colocar sus manos en posiciones determinadas, es necesario reevaluar el tiempo entre cada pose, ya que a veces este tiempo es demasiado largo y en otras ocasiones, dependiendo de la pose, resulta un poco corto.}
    \item {En la prueba de asociación de objetos en necesario ajustar la posición de cada objeto y zona para que el jugador llegue cómodamente a todas ellas. Además, actualmente, no hay forma de recuperar un objeto en caso de que cayera a un lugar inaccesible.}
    \item{La elección de canciones para las pruebas de baile y adivinar la canción, parece ser muy buena.}
    \item{Los sonidos usados en la prueba de asociación de sonidos son claros y fácilmente distinguibles, lo cual es bueno.}
    \item{La posición aleatoria de la fuente de sonido en la prueba donde hay que localizarla puede ser problemática. Hay zonas en las que es muy difícil distinguir la procedencia del sonido.}
\end{itemize}


