\section{Entrega 5}



\subsection{Objetivo}

Esta última entrega del proyecto tiene como objetivo unir todo el trabajo realizado en las anteriores para formar un prototipo de juego completo y jugable, para posteriormente realizar una serie de pruebas con gente real para evaluar la funcionalidad y usabilidad de este proyecto.





\subsection{Iteración 1}

El objetivo inicial de esta entrega era el de diseñar y crear todo el flujo que seguirá el juego. Teniendo en cuenta de que se pretende recrear un concurso televisivo, se va desarrollar una introducción que describa el concurso y presente las reglas al jugador, además de las fases que ocurrirán al principio y final de cada prueba, así como entre cada una de ellas. Finalmente, se añadirá un menú en el que se podrán cambiar los ajustes del juego, entre ellos: la dificultad, los tipos de pruebas que aparecerán, si el jugador está sentado o de pie, o si el jugador tendrá ayuda de una persona externa o no.

Este objetivo se ha visto alterado por pasar a disponer de un nuevo dispositivo de RV para el desarrollo, en este caso las gafas Meta Quest 2. Como se explica en la sección \textbf{Tecnología a usar} (\ref{sec:tecnologiaUsar}), este dispositivo es mucho más deseable para este proyecto, ya que permite prescindir de un ordenador externo al que conectar las gafas, del uso de estaciones base separadas para realizar el seguimiento del casco y los mandos, y al ser más reciente permite utilizar las bibliotecas más modernas para facilitar el desarrollo. Por estas razones, aún suponiendo un extra de carga de trabajo, se decide realizar el cambio de dispositivo, el cual se describe a continuación.


\subsubsection{Migración dispositivo VR}

El primer paso es crear una copia de seguridad en caso de haber alguna incompatibilidad. A continuación, se descarga la última versión del editor de Unity, en este caso se usará la versión 2023.1.9f1. Una vez instalada, abrimos el proyecto usando la nueva versión. Durante la migración de una versión a otra aparecen errores de y se recibe el aviso de que el proyecto puede no funcionar. Cuando se abre el proyecto aparece un nuevo aviso sobre varios paquetes que están obsoletos (\ref{fig:deprecated})




\begin{figure}
	\centering
	\includegraphics[width=0.7\textwidth]{04.Desarrollo/05.Entrega5/01.Iteracion5_1/00.Figuras/03.deprecated.png}
	\caption{Aviso sobre paquetes obsoletos.}
	\label{fig:deprecated}
\end{figure}


Tras desinstalar los paquetes obsoletos, en este caso tanto SteamVR, Zinnia y VRTK, se va a proceder a instalar las nuevas bibliotecas para Meta Quest 2, para ello se va a seguir la documentación oficial. \cite{DES_5_1_oculusSDK} El proceso a seguir es sencillo y consiste en importar el nuevo paquete OculusIntegration (versión 55 en este caso). No ha surgido ningún nuevo problema, la documentación es buena y muy visual, y un proceso similar ya se ha descrito en la \textbf{Entrega 1} (\ref{sec:Entrega1Iteracion1}), por lo que no voy a desarrollar mucho esta instalación.

Una vez instalada la nueva biblioteca, el siguiente paso es recuperar todas las funcionalidades que ya estaban desarrolladas para las HTC Vive anteriormente. Por suerte el SDK de integración de Oculus es muy completo y tiene ya implementado casi todo lo necesario. Usando el objeto OculusInteractionSampleRig (figura \ref{fig:sample_rig}), podemos añadir a la escena tanto la cámara como los controladores y todo lo necesario para el seguimiento y la interacción básica con el juego.

\begin{figure}
	\centering
	\includegraphics[width=0.7\textwidth]{04.Desarrollo/05.Entrega5/01.Iteracion5_1/00.Figuras/04.sample_rig.png}
	\caption{Jerarquía del objeto OculusInteractionSampleRig incluido con el SDK de Meta.}
	\label{fig:sample_rig}
\end{figure}


Para facilitar el uso del juego a personas con pocos conocimientos tecnológicos, se va a usar el seguimiento de manos proporcionado por el visor, de esta forma se proyecta en el juego una representación exacta de las manos de los usuarios y sus movimientos, como se puede ver en la figura \ref{fig:hand_tracking}.

\begin{figure}
	\centering
	\includegraphics[width=0.7\textwidth]{04.Desarrollo/05.Entrega5/01.Iteracion5_1/00.Figuras/05.hand_tracking.png}
	\caption{Manos virtuales captadas usando el seguimiento de Meta Quest 2.}
	\label{fig:hand_tracking}
\end{figure}

A continuación es necesario añadir las posibles interacciones de las manos con el entorno, en este caso: coger y pulsar objetos y poder seleccionar elementos lejanos mediante un rayo. De nuevo, el SDK proporciona elementos ya configurados para estos casos. Solo es necesario añadirlos a la jerarquía según la figura \ref{fig:interactors}. En este caso se añaden tanto para las manos como para los mandos, en caso de desear utilizarlos.


\begin{figure}
	\centering
	\includegraphics[width=0.7\textwidth]{04.Desarrollo/05.Entrega5/01.Iteracion5_1/00.Figuras/06.interactors.png}
	\caption{\textit{Interactors} que permiten coger, pulsar o lanzar rayos.}
	\label{fig:interactors}
\end{figure}

Finalmente, para que un objeto del juego pueda interactuar con el jugador, necesita nuevos scripts también proporcionados por el SDK, en este caso son: 

♠\begin{itemize}
	\item {\textit{Grabbable}: Permite que un objeto pueda ser cogido y define su comportamiento.}
	\item{\textit{Grab Interactable}: Permite que el elemento sea cogido con los mandos.}
	\item{\textit{Hand Grab Interactable}: Permite que el elemento sea cogido con las manos.}
	\item{\textit{Ray Interactable}: Permite lanzar un rayo sobre el objeto para interactuar con él.}
\end{itemize}


Tras estos pasos, la migración queda terminada y podemos comenzar a utilizar Meta Quest 2 y su seguimiento de manos en el proyecto.

















\subsection{Iteración 2}

En esta sección se van a llevar a cabo pruebas con usuarios reales para analizar la eficacia y usabilidad del sistema desarrollado. Para esto se ha creado un cuestionario del que se hablará a continuación y se ha presentado a los usuarios con el juego para que realicen 5 pruebas cada uno. Los usuarios varían tanto en edad y sexo, como en conocimientos tecnológicos previos para obtener información más amplia.



\subsubsection{Cuestionario}

Para evaluar la usabilidad del sistema se ha optado por utilizar un cuestionario SUS (System Usability Scale) ya que se trata de un sistema estándar el el desarrollo de software para medir la calidad y usabilidad de un programa. Se entiende como usabilidad la capacidad del programa para ser comprendido y utilizado por una persona.

Esta es una métrica fundamental en este proyecto, ya que un sistema poco usable, aunque tuviera una gran efectividad, reduciría drásticamente su calidad al dificultar su uso y entendimiento para el usuario. Por desgracia, para este proyecto no se dispone de los conocimientos, profesionales, tiempo ni usuarios para poder hacer un buen análisis de su efectividad y eficacia como sistema contra el deterioro cognitivo leve o para el entrenamiento cognitivo en general.

Por estos motivo, para el ámbito de este proyecto, solo se va a estudiar la usabilidad y todas las opiniones o pensamientos que ofrezcan los participantes.

El cuestionario que se les ha presentado a los usuarios se encuentra detallado en el apéndice \textbf{Cuestionarios} (\ref{sec:apendice:Custionarios}). Se trata de un cuestionario estándar con 10 preguntas a las que se responde con un número entre el 1 y el 5 indicando el grado con el que los usuarios están o no de acuerdo con el enunciado, significando un 1 que están totalmente en desacuerdo, y un 5 que están totalmente de acuerdo. Este método de evaluación se conoce como la escala Linkert y permite obtener un resultado más preciso de las opiniones del usuario al ofrecerle poder expresar su grado exacto de concordancia con los enunciados, además de permitir cuantificar numéricamente las respuestas de cada usuario para posterior análisis \cite{DES_5_2_linkert}.


El cuestionario, siguiendo el estándar SUS contiene diez enunciados, cinco de los cuales son positivos y el resto negativos. Dichos enunciados se presentan de forma alterna para hacer el test lo más homogéneo posible y limitar el sesgo. Una vez respondidos los cuestionarios, se puede obtener una calificación entre 0 y 100 puntos mediante una fórmula matemática sencilla:

\begin{itemize}
	\item{Sumar los resultados de los enunciados positivos y restar 5.}
	
	\item{Restar a 25 la suma de los enunciados negativos.}
	
	\item{Por último, sumar los dos números anteriores y multiplicar el resultado por 2,5.}
	
\end{itemize}

De esta forma, se obtiene un resultado que puede puntuar la usabilidad del sistema según el usuario. Aunque la escala es lineal entre 0 y 100, solo se considera que un sistema tiene una buena usabilidad si su puntuación en el test SUS supera los 68 puntos.

\subsubsection{Pruebas}

Las pruebas que se han realizado consisten en una sesión de juego de una duración de 5 minutos aproximandamente. Las personas seleccionadas para llevar a cabo estas pruebas pertenecen a distintos grupos de edad, conocimientos tecnológicos y familiaridad con la realidad virtual.

Un total de seis personas han prestado su ayuda para este estudio, las cuales se pueden clasificar de la siguiente forma:

\begin{itemize}
	\item{\textbf{Edad}. Tres personas tienen entre 20 y 30 años. Otras 3 tienen más de 60 años.}
	
	\item{\textbf{Conocimientos tecnológicos}. Cuatro personas tienen al menos conocimientos tecnológicos básicos, dos de las cuales tienen conocimientos avanzados. Otras dos personas no tienen conocimientos tecnológicos.}
	
	\item{\textbf{Familiaridad RV}. Dos personas están familiarizadas con la realidad virtual por experiencias previas. Para otras cuatro personas, esta ha sido su primera experiencia en realidad virtual.}
	
\end{itemize}

A cada persona se le ha asignado una letra (de la A a la F) para diferenciarlos y proteger su identidad. Para ver como se distribuyen en concreto estas personas en los tres grupos anteriores, se ha creado la tabla de la figura \ref{fig:tablaPersonasLetras} en la que se pueden ver cada una de las personas encuestadas clasificadas en las tres categorías, según edad, conocimiento tecnológico y si están familiarizadas con la RV o no.


%\begin{figure}
%	\centering
%	\includegraphics[width=0.7\textwidth]{04.Desarrollo/05.Entrega5/02.Iteracion5_2/00.Figuras/01.tabla_personas.png}
%	\caption{Número de personas encuestadas divididas por grupos de edad, conocimientos tecnológicos y familiaridad con la RV.}
%	\label{fig:tablaPersonas}
%\end{figure}


\begin{figure}
	\centering
	\includegraphics[width=0.7\textwidth]{04.Desarrollo/05.Entrega5/02.Iteracion5_2/00.Figuras/02.tabla_personas_letras.png}
	\caption{Personas encuestadas identificadas por letras y divididas por grupos de edad, conocimientos tecnológicos y familiaridad con la RV.}
	\label{fig:tablaPersonasLetras}
\end{figure}


A todas las personas se les ha dado una explicación de la motivación y objetivo del juego, así como de los controles y todo lo que necesitan saber para poder utilizar el juego. En todo momento durante las pruebas las personas han estado acompañadas y, en caso de ser necesario, aconsejadas sobre como actuar o avanzar en el juego.

Aunque el juego y las gafas de RV permiten el uso de forma completamente independiente, las pruebas se han realizado con un cable uniendo las gafas a un ordenador desde el que poder monitorizar lo que ve y hace el jugador, así como poder realizar grabaciones de vídeo para posterior estudio.


En el repositorio de Github en el que se encuentra este proyecto (\url{https://github.com/Eltrio723/TFG}), están disponibles algunos vídeos de las pruebas realizadas.


\subsubsection{Resultados}




\ref{sec:apendice:Custionarios:Resultados}








\subsubsection{Opinión profesional}
















\subsection{Conclusiones de la entrega 5}



En esta última entrega se ha realizado la migración de HTC Vive a Meta Quest 2 para poder aprovechar las ventajas que este dispositivo ofrece, a continuación se han integrado todos los componentes del juego para crear una estructura de clases capaz de darle un flujo al juego y finalmente se han realizado pruebas con personas para evaluar el sistema desarrollado. 

Durante las entregas anteriores se han ido desarrollando componentes de forma individual e independiente para en esta entrega poder crear la estructura superior que lo une todo. Por una parte, esto ha sido una gran ventaja, ya que permitió poder comenzar rápidamente el desarrollo y aprender sobre Unity antes de intentar crear una gran estructura y ralentizar el comienzo del desarrollo. Por otra parte, al realizar el trabajo de esta forma, no ha existido una idea clara de como aunar todas las pruebas hasta el final, provocando, inevitablemente, tener que realizar cambios en componentes ya terminados.

Aún así, pienso que esta estrategia ha sido la adecuada para este proyecto en concreto por su estructura de entregas y el tiempo para realizarlas. Ha permitido tener desde muy pronto la base de las pruebas que son la parte fundamental del juego, y posteriormente unir los elementos como un puzle.

Las pruebas se han hecho con un grupo reducido de personas, que por una parte puede hacer que los resultados no muestren por completo la realidad, pero permite que cada usuario haya podido tener más tiempo para probar los aspectos del juego y proporcionar su opinión además del cuestionario realizado. 

Los resultados en los cuestionarios han sido bastante consistentes, sin haber una destacada diferencia entre usuarios. Dando una puntuación media de 79 puntos en los cuestionarios SUS, los jugadores piensan que el juego tiene una buena usabilidad, a lo que añaden sus buenas impresiones sobre la realidad virtual en general.

Durante las pruebas, ha quedado constatado que desafíos iniciales como la orientación dentro de la RV y la buena visión, incluso utilizando gafas bajo en visor de RV, han sido fácilmente superados por los usuarios y no han supuesto problema en ningún caso. Además, se ha comprobado que el cambio de dispositivo de RV desde el HTC Vive con sus mandos, sus estaciones de seguimiento y la necesidad de cable conector, a Meta Quest 2 sin necesidad de mandos, estaciones ni cables ha sido un gran acierto. Todos los usuarios se han sentido cómodos y no han tenido problema en comprender los controles, como probablemente hubieran tenido usando mandos físicos.

Finalmente, la experta Carmen Granero ha expuesto su opinión sobre como este proyecto tiene una buena base y solo sería necesario la colaboración con profesionales de la cognición para crear un juego que podría ayudar a mucha gente con o sin deterioro cognitivo.
