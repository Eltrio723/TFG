
\subsection{Iteración 1}

El objetivo inicial de esta entrega era el de diseñar y crear todo el flujo que seguirá el juego. Teniendo en cuenta de que se pretende recrear un concurso televisivo, se va desarrollar una introducción que describa el concurso y presente las reglas al jugador, además de las fases que ocurrirán al principio y final de cada prueba, así como entre cada una de ellas. Finalmente, se añadirá un menú en el que se podrán cambiar los ajustes del juego, entre ellos: la dificultad, los tipos de pruebas que aparecerán, si el jugador está sentado o de pie, o si el jugador tendrá ayuda de una persona externa o no.

Este objetivo se ha visto alterado por pasar a disponer de un nuevo dispositivo de RV para el desarrollo, en este caso las gafas Meta Quest 2. Como se explica en la sección \textbf{Tecnología a usar} (\ref{sec:tecnologiaUsar}), este dispositivo es mucho más deseable para este proyecto, ya que permite prescindir de un ordenador externo al que conectar las gafas, del uso de estaciones base separadas para realizar el seguimiento del casco y los mandos, y al ser más reciente permite utilizar las bibliotecas más modernas para facilitar el desarrollo. Por estas razones, aún suponiendo un extra de carga de trabajo, se decide realizar el cambio de dispositivo, el cual se describe a continuación.


\subsubsection{Migración dispositivo VR}









