%%%%%%%%%%%%%%%%%%%%%%%%%%%%%%%%%%%%%%%%%%%%%%%%%%%%%%%%%%%%%%%%%%%
%%% Documento LaTeX 																						%%%
%%%%%%%%%%%%%%%%%%%%%%%%%%%%%%%%%%%%%%%%%%%%%%%%%%%%%%%%%%%%%%%%%%%
% Título:		Introducción
% Autor:  	Ignacio Moreno Doblas
% Fecha:  	2014-02-01, actualizado 2019-11-11
% Versión:	0.5.0
%%%%%%%%%%%%%%%%%%%%%%%%%%%%%%%%%%%%%%%%%%%%%%%%%%%%%%%%%%%%%%%%%%%
% !TEX root = A0.TFG.tex

\chapterbegin{Introducción}
\label{chp:Introduccion}

\section{Introducción}


El ser humano está en constante desarrollo desde que nace. Este proceso no es únicamente físico y biológico, si no que da forma a cada persona desde temprana edad mediante cambios psicológicos y cognitivos. La cognición es la capacidad de percibir información del entorno para posteriormente procesarla y obtener conocimientos y valoraciones. Esto engloba procesos cognitivos como el razonamiento, la memoria, la atención, la capacidad de resolución de problemas y los sentimientos.

 Durante la vida de una persona, la cognición se desarrolla rápidamente durante sus primeros años de vida y, en condiciones normales, se mantiene estable durante la vida adulta. Ciertos motivos externos pueden causar un deterioro anormal de las capacidades cognitivas de un individuo, como son por ejemplo algunas enfermedades neurodegenerativas como el Alzheimer u otros tipos de demencia. Incluso en personas sanas, existe un deterioro natural de las habilidades cognitivas que va ligado a la edad, por lo que es común que las personas de la tercera edad sufran de problemas cognitivos como pérdidas de memoria, dificultades de comunicación o faltas de atención. Estos son problemas que las personas mayores sufren en su día a día y que en ocasiones no les permite disfrutar plenamente de su vida y ocio diario.
 
Afortunadamente, la cognición es una característica humana y como muchas otras, puede ser entrenada, reforzada e incluso recuperada después de verse deteriorada. Para esto existen los llamados entrenamientos cognitivos, que suelen venir asociados a ejercicios mentales. Cualquier persona puede realizar estos ejercicios y ver reforzadas sus capacidades cognitivas, pero son especialmente útiles para prevenir un deterioro anticipado de la cognición en personas mayores y como método de rehabilitación para personas que a causa de alguna enfermedad han visto sus capacidades cognitivas dañadas.

Estos entrenamientos existen desde hace décadas, pero han ido evolucionando con el avance de las nuevas tecnologías. Ahora en vez de tratarse de ejercicios en papel que deben realizarse de manera supervisada, se puede acceder a ellos mediante aplicaciones móviles o páginas web, que permiten un uso más cómodo y atractivo.

Sin embargo, esto tiene un problema asociado: y es que, aunque los ejercicios hayan evolucionado con la tecnología, los principales usuarios de estos ejercicios cognitivos son personas mayores que suelen no estar familiarizados en absoluto con dichas tecnologías. Por tanto, esto genera una barrera adicional para su uso ya que la mayoría de las personas mayores no sabe cómo utilizar estas aplicaciones o páginas web.

Para solventar este problema es necesario facilitar al máximo su uso para los mayores, pero es aún más importante, si cabe, que las personas estén motivadas a realizar los ejercicios. Una falta de motivación puede llegar a reducir en gran medida la efectividad de los entrenamientos cognitivos. 



\section{Motivación}

Hoy en día existen tecnologías como la realidad virtual o RV, que mediante el uso de un visor especial permiten sumergir al usuario en un mundo virtual con un grado de inmersión que ninguna otra tecnología puede alcanzar. Incluso para personas jóvenes y con experiencia tecnológica, la realidad virtual provoca sensaciones extraordinarias y permite vivir experiencias imposibles en la vida real. La RV es una herramienta que usada de forma adecuada puede incrementar de manera considerable la motivación de una persona, especialmente una persona mayor, ya que le abre las puertas a un mundo virtual con el que nunca ha tenido contacto, y el factor sorpresa puede motivarlo a continuar avanzando y descubriendo cosas nuevas.

Es por este motivo, que este proyecto trata de demostrar un posible uso de la realidad virtual aplicada a la realización de ejercicios cognitivos enfocados a personas mayores. Se busca resolver los dos problemas planteados: la falta de motivación y la desconexión tecnológica de las personas mayores.


\section{Objetivo}

El objetivo principal de este proyecto es crear un prototipo de un videojuego que pueda utilizarse para comprobar la efectividad del uso de la realidad virtual como herramienta para el entrenamiento cognitivo en personas mayores.

Para alcanzar este objetivo se va a desarrollar un juego serio\footnote{Juego que tiene un propósito principal no recreativo.}, dentro del cual se presentarán al jugador distintos ejercicios cognitivos basados en ejercicios ya existentes y adaptados al uso de la RV.


\subsection{Objetivos específicos}

Para considerar cumplido el objetivo principal de este proyecto, es necesario cumplir a su vez varios subobjetivos que permitirán el desarrollo del videojuego de forma exitosa:


\begin{itemize}
	\item{Creación de un entorno de realidad virtual especialmente diseñado para ser utilizado por personas mayores.}

	\item{Evaluación de dicho entorno para comprobar que una persona mayor y sin conocimientos tecnológicos sea capaz de interactuar con él de manera cómoda.}

	\item{Diseñar ejercicios de entrenamiento cognitivos adaptados a la realidad virtual a partir de ejercicios tradicionales, manteniendo su funcionamiento central inalterado, asegurando que son efectivos.}

	\item{Diseñar ejercicios cognitivos completamente nuevos, dando pleno uso a las capacidades de la RV de las que carecen los métodos tradicionales.}

	\item{Obtener un videojuego final que sea atractivo y motivador para las personas mayores.}

	\item{El videojuego podrá ser usado de forma personal, o como herramienta guiada por profesionales.}

	\item{Reducir al mínimo las dificultades e inconvenientes que conlleva la RV, especialmente a nivel físico: incomodidades con el casco, posibles cables que limiten el movimiento del usuario, etc.}


\end{itemize}


\section{Estructura de la memoria}

En primer lugar, se va a estudiar la cognición humana y cómo esta puede ser entrenada y desarrollada. Para ello se utilizan diferentes tipos de ejercicios cognitivos, que han ido cambiando con la evolución de las tecnologías. 

Este proyecto busca desarrollar un videojuego de realidad virtual para continuar adaptando estos ejercicios a nuevas tecnologías. Por ello se va a hablar sobre la realidad virtual, su historia y la importancia que tiene hoy en día. Posteriormente se estudiarán cuatro motores de videojuegos, que son los conjuntos de software y tecnologías utilizadas para la creación de videojuegos.

Una vez estudiados estos conceptos fundamentales para el proyecto, se detallará el conjunto de herramientas software y hardware que serán utilizadas para el desarrollo del videojuego, así como la metodología que se seguirá durante el progreso de este TFG.

A continuación, se presenta un diario de desarrollo del videojuego en el que se cuenta paso a paso, y con el apoyo de figuras, todo lo relacionado al diseño y creación del videojuego para la consecución del objetivo del proyecto. Finalmente se cierra esta memoria con unas conclusiones, pensamientos finales y posibles líneas futuras que puede seguir el proyecto.





\chapterend
