% !TEX root = A0.TFG.tex

\chapterbeginx{Conclusiones y líneas futuras}

\todo[inline, size=\tiny]{Conclusiones y líneas futuras}

%Después de todo el desarrollo del proyecto, es pertinente hacer una
%valoración final del mismo, respecto a los resultados obtenidos, las
%expectativas o el resultado de la experiencia acumulada.

%Esta sección es indispensable y en ella se ha de reflejar, lo más
%claramente posible, las aportaciones del trabajo con unas conclusiones
%finales.

%Además, considerando también el estado de la técnica, se deben indicar
%las posibles líneas futuras de trabajo, proponer otros puntos de vista
%o cualquier otra sugerencia como postámbulo del presente trabajo, para
%ser considerada por el lector o el tribunal evaluador.






La realidad virtual es una tecnología muy extendida hoy en día que que ya se utiliza en muchas áreas para crear juegos serios, experiencias de entrenamiento o simulaciones de situaciones peligrosas. Por supuesto también tiene un gran componente social y de diversión para videojuegos más usuales. No hay motivo por el que estos dos aspectos de la realidad virtual no puedan unirse en una misma experiencia, creando un videojuego divertido y que aproveche las cualidades de la realidad virtual para tratar un problema al que se enfrentan todas las personas como es el deterioro cognitivo.

Por esta razón, este proyecto comenzó su desarrollo con el objetivo de crear un juego que usara realidad virtual como apoyo para la realización de ejercicios de entrenamiento y rehabilitación cognitiva. Para conseguir esto era imprescindible alcanzar una serie de objetivos específicos más detallados. 

Este juego puede ser utilizado por cualquier persona que lo desee, pero está especialmente pensado para personas mayores, ya que son el grupo donde el deterioro cognitivo es más común. La mayoría de estas personas no han tenido contacto con la realidad virtual y el uso de la tecnología moderna en sus vidas diarias es limitado. Por esto es imprescindible que la realidad virtual no suponga una barrera a la hora de utilizar el juego. Todo el desarrollo ha estado enfocado a conseguir esto. Se ha elegido el dispositivo de realidad virtual más ligero y sencillo de utilizar, ya que no es necesario el uso de mandos. Que la persona utilice sus propias manos para interactuar es una experiencia muy similar a la vida real y elimina la gran barrera de aprender un esquema de botones y sus funciones.

El entorno virtual se ha creado para ser sencillo, agradable y tranquilo, para minimizar todas aquellas sensaciones que pueden causar mareo o reacciones adversas por la desconexión con el mundo real. Esto ha quedado comprobado durante las pruebas realizadas dónde todos los participantes afirman haberse sentido cómodos y seguros utilizando este juego.

Otro aspecto importante es que las pruebas fueran lo más estimulantes posible para las distintas habilidades cognitivas a las que están enfocadas. Para esto, todas las pruebas han sido diseñadas específicamente para ser integradas en este juego y basándose en todo momento en ejercicios cognitivos ya existentes, que llevan años siendo utilizados y que tienen una base teórica que apoya su eficacia.

Además de estimulantes, las pruebas deben ser accesibles para la mayor cantidad de personas posibles, esto incluye gente en silla de ruedas o con movilidad articular reducida. Por ello, todas las pruebas pueden realizarse tanto de pie como sentado y ninguna requiere de grandes movimientos rápidos. Además, si por algún motivo alguna prueba no fuera adecuada, se puede saltar en cualquier momento.

Durante todo el proyecto se ha pretendido reducir los inconvenientes de la realidad virtual, como el visor que es necesario utilizar o la desorientación que puede causar, a la vez que se potencian las ventajas que ofrece la realidad virtual como herramienta completamente única y novedosa. Se ha intentado sacar el máximo partido a esta tecnología aprovechando capacidades como la de utilizar unicamente las manos para controlar todo de forma natural, la inmersión en el mundo de juego para crear situaciones no habituales y la interacción con objetos y de formas que no serían posibles sin esta tecnología. El jugador se encuentra en mitad de un gran escenario de televisión, rodeado de público y con todo tipo de objetos que aparecen y desaparecen según avanza el juego. Según los usuarios encuestados, han tenido una buena sensación estando en ese entorno, en especial, aquellas personas que nunca habían utilizado la realidad virtual, que han sorprendidas por la experiencia.

Además, este videojuego se ha creado tanto para un uso individual, como para servir de herramienta a profesionales dando tratamiento a otras personas. El juego contiene varias pruebas que animan al usuario a rememorar momentos de su vida y dan pie a una conversación con la persona fuera del juego, que puede hablar con ellos, realizar preguntas y evaluar en general su estado y como están avanzando en las pruebas. Aunque ninguna de las personas con las que se ha realizado el estudio ha tenido dificultad en manejar el juego por si misma una vez explicado el funcionamiento, no todas las personas podrán hacerlo, y en ese caso, una persona puede supervisar constantemente y prestar su ayuda.




%Después de todo el desarrollo del proyecto, es pertinente hacer una
%valoración final del mismo, respecto a los resultados obtenidos, las
%expectativas o el resultado de la experiencia acumulada.

%Esta sección es indispensable y en ella se ha de reflejar, lo más
%claramente posible, las aportaciones del trabajo con unas conclusiones
%finales.

%Además, considerando también el estado de la técnica, se deben indicar
%las posibles líneas futuras de trabajo, proponer otros puntos de vista
%o cualquier otra sugerencia como postámbulo del presente trabajo, para
%ser considerada por el lector o el tribunal evaluador.







\chapterend
