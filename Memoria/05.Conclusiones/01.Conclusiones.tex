% !TEX root = A0.TFG.tex

\chapterbeginx{Conclusiones y líneas futuras}



%Después de todo el desarrollo del proyecto, es pertinente hacer una
%valoración final del mismo, respecto a los resultados obtenidos, las
%expectativas o el resultado de la experiencia acumulada.

%Esta sección es indispensable y en ella se ha de reflejar, lo más
%claramente posible, las aportaciones del trabajo con unas conclusiones
%finales.

%Además, considerando también el estado de la técnica, se deben indicar
%las posibles líneas futuras de trabajo, proponer otros puntos de vista
%o cualquier otra sugerencia como postámbulo del presente trabajo, para
%ser considerada por el lector o el tribunal evaluador.






La realidad virtual es una tecnología muy extendida hoy en día que que ya se utiliza en muchas áreas para crear juegos serios, experiencias de entrenamiento o simulaciones de situaciones peligrosas. Por supuesto también tiene un gran componente social y de diversión para videojuegos más usuales. No hay motivo por el que estos dos aspectos de la realidad virtual no puedan unirse en una misma experiencia, creando un videojuego divertido y que aproveche las cualidades de la realidad virtual para tratar un problema al que se enfrentan todas las personas como es el deterioro cognitivo.

Por esta razón, este proyecto comenzó su desarrollo con el objetivo de crear un juego que usara realidad virtual como apoyo para la realización de ejercicios de entrenamiento y rehabilitación cognitiva. Para conseguir esto era imprescindible realizar un estudio sobre la realidad virtual y la cognición humana para aprender sobre dichos temas y poder desarrollar una aplicación capaz de alcanzar los objetivos específicos planteados para este proyecto. 

Este juego puede ser utilizado por cualquier persona que lo desee, pero está especialmente pensado para personas mayores, ya que son el grupo donde el deterioro cognitivo es más común. La mayoría de estas personas no han tenido contacto con la realidad virtual y el uso de la tecnología moderna en sus vidas diarias es limitado. Por esto es imprescindible que la realidad virtual no suponga una barrera a la hora de utilizar el juego. Todo el desarrollo ha estado enfocado a conseguir esto. Se ha elegido el dispositivo de realidad virtual más ligero y sencillo de utilizar, ya que no es necesario el uso de mandos. Que la persona utilice sus propias manos para interactuar es una experiencia muy similar a la vida real y elimina la gran barrera de aprender un esquema de botones y sus funciones.

El entorno virtual se ha creado para ser sencillo, agradable y tranquilo, para minimizar todas aquellas sensaciones que pueden causar mareo o reacciones adversas por la desconexión con el mundo real. Esto ha quedado comprobado durante las pruebas realizadas dónde todos los participantes afirman haberse sentido cómodos y seguros utilizando este juego.

Otro aspecto importante es que las pruebas fueran lo más estimulantes posible para las distintas habilidades cognitivas a las que están enfocadas. Para esto, todas las pruebas han sido diseñadas específicamente para ser integradas en este juego y basándose en todo momento en ejercicios cognitivos ya existentes, que llevan años siendo utilizados y que tienen una base teórica que apoya su eficacia.

Además de estimulantes, las pruebas deben ser accesibles para la mayor cantidad de personas posibles, esto incluye gente en silla de ruedas o con movilidad articular reducida. Por ello, todas las pruebas pueden realizarse tanto de pie como sentado y ninguna requiere de grandes movimientos rápidos. Además, si por algún motivo alguna prueba no fuera adecuada, se puede saltar en cualquier momento.

Durante todo el proyecto se ha pretendido reducir los inconvenientes de la realidad virtual, como el visor que es necesario utilizar, o la desorientación que puede causar. A la vez que se han intentado potenciar las ventajas que ofrece la realidad virtual como herramienta completamente única y novedosa. Se ha intentado sacar el máximo partido a esta tecnología aprovechando capacidades como la de utilizar unicamente las manos para controlar todo de forma natural, la inmersión en el mundo de juego para crear situaciones no habituales, o la interacción con objetos de formas que no serían posibles sin esta tecnología. El jugador se encuentra en mitad de un gran escenario de televisión, rodeado de público y con todo tipo de objetos que aparecen y desaparecen según avanza el juego. Según los usuarios encuestados, han tenido una buena sensación estando en ese entorno, en especial, aquellas personas que nunca habían utilizado la realidad virtual, que han quedado sorprendidas por la experiencia.

Además, este videojuego se ha creado tanto para un uso individual, como para servir de herramienta a profesionales dando tratamiento a otras personas. El juego contiene varias pruebas que animan al usuario a rememorar momentos de su vida y dan pie a una conversación con la persona fuera del juego, que puede hablar con ellos, realizar preguntas y evaluar en general su estado y como están avanzando en las pruebas. Una vez explicado el funcionamiento, ninguno de los usuarios con los que se ha realizado el estudio ha tenido dificultad en manejar el juego por sí mismo. Aunque no todas las personas podrían hacerlo, y en ese caso, sería necesaria la ayuda de un supervisor.




\vspace{5mm}



Se ha creado un prototipo funcional de un juego que, en general, ha conseguido alcanzar sus objetivos. Aún así, llegar hasta este resultado no ha sido fácil. El desarrollo de videojuegos es una materia más compleja de lo que parece a priori, incluso más si se tiene en cuenta el uso de la realidad virtual. Programar objetos dentro de un entorno virtual donde tienen que interactuar con todo el resto de sistemas que incluye el motor de desarrollo es complejo. A pesar de existir una gran cantidad de documentación de calidad para Unity, los recursos se reducen cuando se trata de realidad virtual, ya que se trata de una tecnología relativamente nueva y no existe un gran estándar asentado. Esto ha supuesto un reto importante, ya que se crean nuevas bibliotecas y recursos constantemente, existen grandes cambios entre versiones y la compatibilidad entre dispositivos es baja. 

Se comenzó este proyecto usando VRTK como paquete principal, sin embargo, se encontraba en un cambio de versión que requería la reescritura de prácticamente todo su código. Esto hacía que la versión anterior tuviera falta de componentes importantes, pero que la nueva versión estuviera en fase beta, sin partes básicas y con una documentación oficial nula. Por suerte este problema se solucionó al migrar el proyecto para utilizar el visor de RV Meta Quest 2, ya que utiliza un SDK muy extenso y pulido (actualmente se encuentra en su versión 55), y que contiene muchos ejemplos ilustrativos y buena documentación.


A estas dificultades se le suma la necesidad de crear la mayoría del arte utilizado en el juego. Es un aspecto vital para los videojuegos de realidad virtual tener una apariencia realista y detallada. Esto ha requerido el aprendizaje de diseño 3D y escultura digital, una técnica difícil de aprender y dominar como tantas otras, y que creo que en cierto modo ha perjudicado al juego en la misma medida que lo ha beneficiado. Sin duda, un escenario simple y poco llamativo puede no llamar la atención o incluso aburrir a la gente, pero en este caso en particular, creo que este estilo ha facilitado la adaptación de las personas que lo han probado y que en su mayoría no han tenido contacto anterior con la realidad virtual. Este escenario les ha permitido experimentar y disfrutar sin hacerles sentir completamente desconectados de la realidad, reduciendo la probabilidad de mareos o desorientaciones.

%A pesar de ya conocer la importancia de la buena documentación en el desarrollo de software, tras este proyecto creo que es fundamental que exista una documentación robusta para cualquier programa que se use. Y es importante tener en cuenta que el desarrollo de software lleva tiempo durante el cual se pueden producir cambios en los proyectos y es necesario gestionarlos. 

\vspace{5mm}

En general, este trabajo ha sido exitoso y demuestra la capacidad de la realidad virtual de ser aplicada en infinidad de campos, incluyendo el entrenamiento cognitivo. Tras las pruebas realizadas, incluso personas de edad avanzada han estado a gusto y no han tenido problemas en adaptarse a esta tecnología. Demostrando que este podría ser un proyecto con un interesante desarrollo futuro.


El primer paso claro para mejorar y ampliar este proyecto sería contar con la colaboración de profesionales en el campo de la cognición humana para revisar los ejercicios integrados en este proyecto y mejorarlos, ampliarlos y desarrollar nuevos ejercicios y pruebas que tengan mayor efectividad y sean más atractivos para los jugadores. En segundo lugar, la colaboración de artistas digitales para crear un entorno más inmersivo también puede incrementar la diversión y la eficacia de todos los juegos.

Este proyecto es un prototipo de juego sobre el que se puede iterar y añadir gran cantidad de nuevas funcionalidades, como la capacidad de personalizar la rutina de pruebas que se presentan al jugador, pudiendo descartar pruebas que no se quieran o puedan realizar por la persona en cuestión. O tener en cuenta otras opciones de accesibilidad como la posibilidad de gran limitación de movimiento en los brazos, sordera o ceguera. También se puede beneficiar de características típicas de los videojuegos tradicionales, como una puntuación para el jugador durante cada partida, mejoras en la interfaz y experiencia de usuario, y distintos modos de juego, por ejemplo un modo en el que se elige directamente la prueba que se quiere realizar, sin seguir el flujo diseñado para el juego normal.

Para facilitar el uso de este juego a personas con deterioro cognitivo, se puede utilizar un dispositivo móvil para tener visión de lo que está viendo el jugador y poder ayudarlo y apoyarlo. Como mejora a este sistema se podría desarrollar una aplicación móvil que se empareje con el juego y permita no solo ver, si no interactuar directamente con el juego, tanto para mejorar el apoyo al jugador, como para crear un sistema similar al multijugador asimétrico en el que ambas personas participan activamente en el juego. Esto puede añadir una capa de juego social en la que un profesional puede interactuar con el juego y el jugador para fomentar y mejorar la efectividad de los ejercicios.


%Después de todo el desarrollo del proyecto, es pertinente hacer una
%valoración final del mismo, respecto a los resultados obtenidos, las
%expectativas o el resultado de la experiencia acumulada.

%Esta sección es indispensable y en ella se ha de reflejar, lo más
%claramente posible, las aportaciones del trabajo con unas conclusiones
%finales.

%Además, considerando también el estado de la técnica, se deben indicar
%las posibles líneas futuras de trabajo, proponer otros puntos de vista
%o cualquier otra sugerencia como postámbulo del presente trabajo, para
%ser considerada por el lector o el tribunal evaluador.







\chapterend
